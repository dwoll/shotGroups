%\ifpdf
\graphicspath{{graphics_pdf/}}
%\graphicspath{{graphics_pdf_sw/}}
%\usepackage[final,babel,activate=true]{microtype}
\usepackage[%pdftex,%
plainpages=false,%
unicode=true,%
pdfpagelabels=true,%
colorlinks=true,%
linkcolor=blue,%
citecolor=blue,%
filecolor=blue,%
urlcolor=blue,%
%colorlinks=false,%
pdftitle={Grundlagen der Datenanalyse mit R},%
pdfauthor={Daniel Wollschläger},%
pdfsubject={Eine anwendungsorientierte Einführung},%
pdfkeywords={R, Psychologie, Statistik, Datenauswertung}%
]{hyperref}
%\pdfinfo{
%   /Author (Daniel Wollschläger)
%   /Title  (Grundlagen der Datenanalyse mit R)
%   /Subject (Eine anwendungsorientierte Einführung)
%   /Keywords (R, Psychologie, Statistik, Datenauswertung)
%}
%\else
%\usepackage{hyperref}
%\graphicspath{{graphics_eps/}}
%\fi

%%%%%%%%%%%%%%%%%%%%%%%%%%%%%%%%%%%%%%%%%%%%%%%%%%%%%%%%%%%%%%%%%%
%%%%%%%%%%%%%%%%%%%%%%%%%%%%%%%%%%%%%%%%%%%%%%%%%%%%%%%%%%%%%%%%%%

% for package listings
\lstloadlanguages{R}

\lstset{%
language=R,%
basicstyle=\ttfamily\upshape\mdseries,%
keywordstyle=\ttfamily\upshape\mdseries,%
commentstyle=\ttfamily\upshape\mdseries,%
stringstyle=\ttfamily\upshape\mdseries,%
identifierstyle=\ttfamily\upshape\mdseries,%
columns=flexible,%
keepspaces=true,%
tabsize=4,%
numbers=none,%
extendedchars=true,%
breaklines=true,%
breakatwhitespace=true,
breakindent=0pt,%
breakautoindent=false,%
prebreak=\makebox[0pt][l]{\tiny$\searrow$},%
postbreak=\makebox[0pt][r]{\tiny$\rightarrow$},%
showspaces=false,%
showstringspaces=false,%
literate={<<}{{$\langle$}}{1}{>>}{{$\rangle$}}{1},%
showtabs=false}

%%%%%%%%%%%%%%%%%%%%%%%%%%%%%%%%%%%%%%%%%%%%%%%%%%%%%%%%%%%%%%%%%%
%%%%%%%%%%%%%%%%%%%%%%%%%%%%%%%%%%%%%%%%%%%%%%%%%%%%%%%%%%%%%%%%%%

\newcommand*\myURL[1]{\texttt{#1}}
\newcommand*{\euler}{{\mathrm{e}}}
%\newcommand*{\kronecker}{\raisebox{1pt}{\ensuremath{\:\otimes\:}}}

% larger vertical table-spacing
\renewcommand{\arraystretch}{1.2}

%%%%%%%%%%%%%%%%%%%%%%%%%%%%%%%%%%%%%%%%%%%%%%%%%%%%%%%%%%%%%%%%%%
%%%%%%%%%%%%%%%%%%%%%%%%%%%%%%%%%%%%%%%%%%%%%%%%%%%%%%%%%%%%%%%%%%

\newindex{default}{idx}{ind}{Index}
\newindex{func}{idxF}{indF}{\textsf{R}-Funktionen, Klassen und Schlüsselwörter}
\newindex{pack}{idxP}{indP}{Zusatzpakete}
